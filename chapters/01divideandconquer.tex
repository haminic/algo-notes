\chapter{Divide-and-Conquer}

\section{ปัญหา Maximum Subarray}

สมมติเรามีลำดับของจำนวนจริง (array) ชุดหนึ่ง และเราต้องการหาลำดับย่อยที่เรียงติดกันที่มีผลรวมมากที่สุด (จะเรียกว่าเป็น \emph{maximum subarray}) เราอาจจะทำตรง ๆ เลยโดยเช็คทุก ๆ ลำดับย่อยที่เป็นไปได้ซึ่งถ้าลำดับนี้มีจำนวนสมาชิกอยู่ $n$ ตัว จะทำให้ต้องเช็คลำดับย่อยทั้งหมด $\binom{n}{2}$ ชุด จึงต้องใช้เวลา $\Omega(n^2)$

อีกวิธีที่ดีกว่าคือการใช้ recursion โดยเราจะแบ่ง array ที่ได้รับมานี้ออกเป็น 2 subarray (โดยจะเก็บ index ไว้สามตัวคือ $\id{low}$, $\id{mid}$, และ $\id{high}$) ไปเรื่อย ๆ และหา maximum subarray ของ subarray ซ้าย, subarray ขวา, และ maximum subarray ที่ข้ามจุดแบ่ง (crossing subarray) จากนั้นเลือกค่าที่มากที่สุดในสามกรณีนี้

สังเกตว่าเราสามารถหา maximum crossing subarray ของ array $A$ ขนาด $n$ ที่ผ่าน $\id{mid}$ ได้โดยการหา maximum subarray ของครึ่งซ้ายรวมกับของครึ่งขวา:
\begin{codebox}{การหา Maximum Crossing Subarray}
	\Function{findMaxCrossingSubarray}{A, \id{low}, \id{mid}, \id{high}}
		\State $\id{leftSum} \gets -\infty$
		\State $\id{sum} \gets 0$
		\For{$i\gets\id{mid}$ \Downto $\id{low}$}
			\State $\id{sum} \gets \id{sum} + A[i]$
			\If{$\id{sum} > \id{leftSum}$}
				\State $\id{leftSum} = \id{sum}$
				\State $\id{maxLeft} = i$
			\EndIf
		\EndFor
		\State ทำคล้ายกันสำหรับฝั่งขวา โดยลูปตั้งแต่ $\id{mid}+1$ ถึง $\id{high}$
		% \State $\id{rightSum} \gets -\infty$
		% \State $\id{sum} \gets 0$
		% \For{$i\gets\id{mid}+1$ \To $\id{high}$}
		%     \State $\id{sum} \gets \id{sum} + A[i]$
		%     \If{$\id{sum} > \id{rightSum}$}
		%         \State $\id{rightSum} = \id{sum}$
		%         \State $\id{maxLeft} = i$
		%     \EndIf
		% \EndFor
		\State \Return $\tuple{\id{maxLeft}, \id{maxRight}, \id{leftSum} + \id{rightSum}}$
	\EndFunction
\end{codebox}
ซึ่งใช้เวลา $\Theta(n)$

ดังนั้นก็จะได้อัลกอริทึมในการหา maximum subarray โดยการ \emph{divide-and-conquer}:
\begin{icodebox}{อัลกอริทึม Maximum Subarray โดยใช้ Divide-and-Conquer}
	\Function{findMaxSubarray}{A, \id{low}, \id{high}}
		\If{$\id{low} = \id{high}$}
			\State \Return $\ab(\id{low}, \id{high}, A[\id{low}])$ \Comment{ขั้นฐานเมื่อ subarray มีขนาดเป็น $1$}
		\EndIf
		\State $\id{mid} \gets \floor{(\id{low} + \id{high})/2}$
		\State $\tup{left}{3} \gets \call{findMaxSubarray}{A, \id{low}, \id{mid}}$
		\State $\tup{mid}{3} \gets \call{findMaxCrossingSubarray}{A, \id{low}, \id{mid}, \id{high}}$
		\State $\tup{right}{3} \gets \call{findMaxSubarray}{A, \id{mid}+1, \id{high}}$
		\State $\id{maxSubarray} \gets$ เลือก $x \in \{\idlist{\id{left}, \id{mid}, \id{right}}\}$ ที่มีค่าของ $x_3$ มากที่สุด
		\State \Return $\tup{maxSubarray}{3}$
	\EndFunction
\end{icodebox}
โดยเราจะเรียก $\call{findMaxSubarray}{A, 1, A.\id{length}}$ เมื่อต้องการหา maximum subarray ของ $A$ 

ถ้ากำหนดให้อัลกอริทึมนี้ทำงานได้ในเวลา $T(n)$ ก็จะได้ความสัมพันธ์
\begin{eqnobox}
	T(n) = T(\floor{n/2}) + T(\ceiling{n/2}) + \Theta(n)
\end{eqnobox}
(เพราะการหา $\id{left}$ และ $\id{right}$ เป็นการเรียกฟังก์ชันเดิมนี้ซ้ำ โดยที่ array มีขนาดลดลงครึ่งหนึ่ง ใช้เวลา $T(\floor{n/2})$, การหา $\id{mid}$ ใช้เวลา $\Theta(n)$, และที่เหลือทั้งหมดใช้เวลา $\Theta(1)$) โดยในส่วนถัด ๆ ไปเราจะแก้ได้ว่าความสัมพันธ์เวียนเกิดนี้มีคำตอบ $T(n) = \Theta(n\lg n)$ ซึ่งเร็วกว่าการทำตรง ๆ

\section{อัลกอริทึมการคูณเมทริกซ์ของ Strassen}

\subsection{การคูณเมทริกซ์จตุรัสโดยใช้ Divide-and-Conquer แบบตรง ๆ}

จากที่ได้เห็นว่าการใช้ divide-and-conquer อาจทำให้ได้อัลกอริทึมที่ไวกว่าการทำปกติ เราลองมาพยายามใช้ divide-and-conquer กับการคูณเมทริกซ์จตุรัส โดยแบ่งเมทริกซ์ $n\times n$ เป็น $4$ เมทริกซ์ที่มีขนาด $n/2\times n/2$ (อาจปัดขึ้น-ปัดลงได้เพื่อให้เป็นจำนวนเต็ม แต่จะละไว้ในฐานที่เข้าใจเพราะจริง ๆ แล้วเราสามารถพิสูจน์ได้ว่าการปัดขึ้น-ปัดลงไม่ส่งผลต่อ asymptotic runtime ของอัลกอริทึม) แล้วค่อยนำเมทริกซ์ย่อยมาคูณกัน ดังนี้
\begin{codebox}{อัลกอริทึมการคูณเมทริกซ์จตุรัสโดยใช้ Divide-and-Conquer}
	\Function{recurMatrixMultiply}{A, B}
		\State $n \gets A.\id{rows}$
		\State ให้ $C$ เป็นเมทริกซ์ใหม่ที่แทนผลลัพธ์ของ $A\times B$
		\If{$n = 1$}
			\State $c_{11} \gets a_{11}b_{11}$ \Comment{ขั้นฐานในกรณี $n=1$ ให้คูณแบบสเกลาร์ปกติ}
		\Else
			\State แบ่งเมทริกซ์ $A$, $B$, และ $C$ ออกเป็น $A_{11}$, $A_{12}$, $A_{21}$, $A_{22}$, \dots\ ที่มีขนาด $n/2 \times n/2$
			\State $C_{11} \gets \call{recurMatrixMultiply}{A_{11}, B_{11}} + \call{recurMatrixMultiply}{A_{12}, B_{21}}$
			\State $C_{12} \gets \call{recurMatrixMultiply}{A_{11}, B_{12}} + \call{recurMatrixMultiply}{A_{12}, B_{22}}$
			\State $C_{21} \gets \call{recurMatrixMultiply}{A_{21}, B_{11}} + \call{recurMatrixMultiply}{A_{22}, B_{21}}$
			\State $C_{22} \gets \call{recurMatrixMultiply}{A_{21}, B_{11}} + \call{recurMatrixMultiply}{A_{22}, B_{21}}$
		\EndIf
		\State \Return $C$
	\EndFunction
\end{codebox}
หมายเหตุ: \emph{การคูณทั้งหมดนี้เป็นการคูณจาก index จึงไม่ต้องเสียเวลาในการคัดลอกข้อมูลมาใส่เมทริกซ์ใหม่ทุกการ call}

โดยถ้าอัลกอริทึมนี้ใช้เวลา $T(n)$ ในการคูณเมทริซ์จตุรัสขนาด $n\times n$ เรามาพิจารณาความสัมพันธ์เวียนเกิดของ $T$ โดยเริมจากการแบ่งเมทริกซ์ ใช้เวลา $\Theta(1)$ (เพราะเป็นการคูณโดยแบ่งจาก index) และถัดมาเราจะต้องเรียกฟังก์ชันนี้ซ้อนไปอีก $8$ รอบ ใช้เวลา $8T(n/2)$ สุดท้าย ในการนำเมทริกซ์ย่อยที่คูณกันแล้วมาบวกกัน ใช้เวลา $\Theta(n^2)$ ดังนั้น
\begin{eqnobox}[label=01:eq:matrixrecursion]
	T(n) = 8\,T(n/2) + \Theta(n^2)
\end{eqnobox}
ซึ่งเมื่อใช้ master theorm (ในส่วนถัดไป) จะได้ว่า $T(n) = \Theta(n^3)$ ซึ่งไม่ต่างอะไรกับการเขียนแบบคูณตรง ๆ จึงไม่ได้มีประโยชน์อะไรมากนัก

\subsection{การคูณเมทริกซ์จตุรัสด้วยอัลกอริทึมของ Strassen}

แต่ Volker Strassen ได้ค้นพบวิธีทั่วไปที่เร็วกว่าแบบทำตรง ๆ ที่มีไอเดียมาจาก divide-and-conquer เช่นกัน โดยสังเกตว่าจาก (\ref{01:eq:matrixrecursion}) การบวกเมทริกซ์ไม่ได้ส่งผลกับความสัมพันธ์เวียนเกิดเท่าไหร่ เพราะไม่วาจะบวกกันกี่ครั้ง ถ้าจำนวนครั้งในการบวกคงที่ สุดท้ายแล้วพจน์หลังก็จะเป็น $\Theta(n^2)$ อยู่ดี แต่ในทางกลับกันจำนวนครั้งในการคูณเมทริกซ์ส่งผลโดยตรงกับสัมประสิทธ์ด้านหน้า $T(n/2)$ ดังนั้นถ้าเราสามารถลดจำนวนครั้งในการคูณเมทริกซ์โดยแลกกับการบวกเมทริกซ์เพิ่มอีกนิด ตามเซนส์แล้วน่าจะทำให้ time complexity สุดท้ายลดลง (เหตุผลจริง ๆ คือจาก master theorem แล้ว $n^{\log_2 8}/n^2 = n^1$ แปลว่ายังสามารถลด $8$ ลงไปได้อีกในขณะที่ยังไม่ทำให้พจน์ $\Theta(n^2)$ โตเร็วเกินไป)

โดยอัลกอริทึมของ Strassen ลดจำนวนการคูณเมทริกซ์ย่อยจาก $8$ ครั้งเหลือเพียง $7$ ครั้ง แลกกับการบวกและลบเมทริกซ์เพิ่มขึ้น ดังนี้:
\begin{icodebox}{อัลกอริทึมการคูณเมทริกซ์จตุรัสของ Strassen}
	\Function{strassenMatrixMultiply}{A, B}
		\State $n \gets A.\id{rows}$
		\If{$n = 1$}
			\State $c_{11} \gets a_{11}b_{11}$
		\Else
			\State แบ่งเมทริกซ์ $A$, $B$, และ $C$ ออกเป็น $A_{11}$, $A_{12}$, $A_{21}$, $A_{22}$, \dots\ ที่มีขนาด $n/2 \times n/2$
			\State $P_1 \gets \call{strassenMatrixMultiply}{A_{11} + A_{22}, B_{11} + B_{22}}$
			\State $P_2 \gets \call{strassenMatrixMultiply}{A_{21} + A_{22},\ B_{11}}$
			\State $P_3 \gets \call{strassenMatrixMultiply}{A_{11}, B_{12} - B_{22}}$
			\State $P_4 \gets \call{strassenMatrixMultiply}{A_{22}, B_{21} - B_{11}}$
			\State $P_5 \gets \call{strassenMatrixMultiply}{A_{11} + A_{12}, B_{22}}$
			\State $P_6 \gets \call{strassenMatrixMultiply}{A_{21} - A_{11}, B_{11} + B_{12}}$
			\State $P_7 \gets \call{strassenMatrixMultiply}{A_{12} - A_{22}, B_{21} + B_{22}}$
			\State $C_{11} \gets P_1 + P_4 - P_5 + P_7$
			\State $C_{12} \gets P_3 + P_5$
			\State $C_{21} \gets P_2 + P_4$
			\State $C_{22} \gets P_1 - P_2 + P_3 + P_6$
		\EndIf
		\State \Return $C$
	\EndFunction
\end{icodebox}
ซึ่งจะได้ความสัมพันธ์เวียนเกิดของเวลาในการทำงานคือ
\begin{eqnobox}
	T(n) = 7\,T(n/2) + \Theta(n^2)
\end{eqnobox}
ซึ่งเมื่อใช้ master theorem จะได้ว่า $T(n) = \Theta(n^{\log_2 7}) \approx \Theta(n^{2.81})$ เร็วกว่าการคูณแบบปกติที่ใช้เวลา $\Theta(n^3)$

การคูณและนำมาประกอบกันแบบนี้ดูเหมือน \emph{black magic} แต่ไอเดียหลัก ๆ คือเมื่อลองพิจารณาจากความสัมพันธ์เวียนเกิด เราสามารถสังเกตได้ว่าการบวก ``คุ้ม'' กว่าการคูณมากในแง่ของเวลาในการทำงาน ดังนั้นหากสามารถลดจำนวนการคูณลงได้ แม้จะต้องเพิ่มจำนวนการบวกขึ้น ก็จะทำให้ time complexity โดยรวมลดลงอย่างมีนัยสำคัญ

\section{ความสัมพันธ์เวียนเกิด}

\subsection{การพิสูจน์ความสัมพันธ์เวียนเกิด}

เราสามารถพิสูจน์คำตอบ (แบบ asymptotic) ของความสัมพันธ์เวียนเกิดได้โดยใช้การอุปนัยอย่างเข้ม (strong induction) ตัวอย่างเช่น

\begin{exbox}
	จงพิสูจน์ว่าความสัมพันธ์เวียนเกิด $T(n) = 2\, T(n/2) + n$ มีคำตอบคือ $T(n) = \Theta(n\lg n)$
\end{exbox}
\begin{proof}
	ก่อนอื่นจะพิสูจน์ว่า $T(n) = O(n\lg n)$ โดยจะแสดงว่ามี $n_0 \in \NN$ และ $c \in \RR^+$ ซึ่งทำให้ $T(n) \leq cn\lg n$ ทุก ๆ จำนวนเต็ม $n > n_0$
	
	\underline{ขั้นอุปนัย}: สมมติ $T(t) \leq ct\lg t$ ทุก ๆ จำนวนเต็ม $n_0 < t < n$ จะได้ว่า
	\[
		T(n) = 2\,T(n/2) + n \leq cn\lg n - \frac{c}{2}n + n
	\]
	ดังนั้นจะเห็นว่าถ้าเลือ $c \geq 2$ ก็จะทำให้สามารถพิสูจน์ได้ว่าขั้นอุปนัยเป็นจริง
	
	\underline{ขั้นฐาน}: สังเกตว่าถ้าเราให้ขั้นฐานเป็น $n = 1$ จะได้ว่า
	\[
		T(1) \leq c\lg 1 = 0
	\]
	ซึ่งไม่เป็นจริงอยู่แล้ว เราจึงลองย้ายขั้นฐานมาเป็น $n = 2$ และ $3$ (เพราะว่าสำหรับความสัมพันธ์เวียนเกิดที่เป็นจำนวนเต็ม ถ้า $T$ ติดเป็น floor function หรือ ceiling function แทนจะทำให้ทุก ๆ ค่า $T(n)$ เขียนได้อยู่ในรูปของ $T(2)$ และ $T(3)$ ได้เสมอ ไม่จำเป็นต้องติด $T(1)$ ทำให้ขั้นอุปนัยไม่เจ๊ง) ก็จะได้
	\[
		T(2) \leq 2c\lg 2 \qquad T(3) \leq 3c\lg 3
	\]
	ดังนั้นถ้าเราเลือก $c$ ที่มีค่ามากพอจะได้ขั้นฐานและขั้นอุปนัยเป็นจริง
	
	โดยหลักอุปนัยเชิงคริตศาสตร์อย่างเข้ม จึงได้ว่า $T(n) = O(n\lg n)$ (การเขียนพิสูจน์จริง ๆ จะต้องเริ่มจากหาค่า $c$ มาเลย แต่การเขียนแบบนี้จะเห็นชัดกว่าว่าค่า $c$ นี้มาจากไหน)
	
	ในทำนองเดียวกันเราจะสามารถแสดงได้ว่า $T(n) = \Omega(n\lg n)$ และเนื่องจาก $T(n) = O(n\lg n)$ และ $T(n) = \Omega(n\lg n)$ ก็จะได้ว่า $T(n) = \Theta(n\lg n)$ ตามต้องการ
\end{proof}

แต่กสังเกตว่าการพิสูจน์คำตอบแบบนี้เราจะต้อง ``เดา'' คำตอบของความสัมพันธ์เวียนเกิดให้ได้ก่อน โดยวิธีในการเดาคำตอบที่ดีวิธีหนึ่งคือการวาด \emph{recursion tree} (หรือบางครั้งถ้าเขียนออกมาดีพอ อาจนับว่าเป็นการพิสูจน์ได้เลย) ยกตัวอย่างเช่น

\begin{exbox}
	จงพิสูจน์ว่าความสัมพันธ์เวียนเกิด $T(n) = 4\,T(n/2) + n^2\lg n$ มีคำตอบคือ $T(n) = O(n^2\lg^2 n)$
\end{exbox}
\begin{proof}
	
\end{proof}